\documentclass[10pt,a4paper]{article}
\usepackage[utf8]{inputenc}
\usepackage[T1]{fontenc}
\usepackage[brazil]{babel}
\usepackage[margin=2.5cm]{geometry}
\usepackage{tcolorbox}
\usepackage{tabularx}
\usepackage{booktabs}
\usepackage{array}
\usepackage{hyperref}
\usepackage{longtable}
\usepackage{multicol}
\usepackage{graphicx}
\usepackage{fancyhdr}
\usepackage{caption}
\usepackage[table]{xcolor} % Adicionado para colorir as tabelas
\usepackage{xcolor}
% Configuração do cabeçalho e rodapé
\pagestyle{fancy}
\fancyhf{} % Limpa os campos de cabeçalho e rodapé
\renewcommand{\headrulewidth}{0pt} % Remove a linha do cabeçalho
\fancyfoot[C]{\thepage} % Número da página no centro do rodapé

% Comando para o cabeçalho vermelho da tabela
\definecolor{vermelhoheader}{HTML}{800000}

% Comando para linha de cabeçalho da tabela
\newcommand{\redrow}{\rowcolor{vermelhoheader}\color{white}\textbf}

\setlength{\parskip}{0.5em}
\setlength{\parindent}{0pt}
\renewcommand{\arraystretch}{1.2}

\title{PreventAI — Entidades e Relacionamentos (MER)}
\author{Equipe Rocket}
\date{\today}

\begin{document}

\fancypagestyle{plain}{\pagestyle{plain}}

\begin{center}
    \large{\textbf{PreventAI}} \\
    \normalsize{Modelo de Entidades e Relacionamentos (MER)} \\
\end{center}
\vspace{1cm}

\section*{Entidades, Atributos e Tipos de Atributos}

\subsection*{ENTIDADE: SITE}
Representa a planta industrial (local físico) onde estão linhas e máquinas.

\begin{table}[h!]
\centering
\begin{tabular}{| l | l | c | c |}
\hline
\rowcolor{red}\color{white}\textbf{Atributo} & \color{white}\textbf{Tipo do Atributo} & \color{white}\textbf{Card. Mínima} & \color{white}\textbf{Card. Máxima} \\
\hline
cd\_site & Simples, determinante & 1 & 1 \\
nm\_site & Simples & 1 & 1 \\
sg\_pais & Simples & 1 & 1 \\
nm\_cidade & Simples & 1 & 1 \\
\hline
\end{tabular}
\caption{Entidade SITE e seus respectivos atributos, tipos e cardinalidade.}
\label{tab:site}
\end{table}

\subsection*{ENTIDADE: LINHA\_PRODUCAO}
Agrupa máquinas por linha/célula dentro de um site.

\begin{table}[h!]
\centering
\begin{tabular}{| l | l | c | c |}
\hline
\rowcolor{red}\color{white}\textbf{Atributo} & \color{white}\textbf{Tipo do Atributo} & \color{white}\textbf{Card. Mínima} & \color{white}\textbf{Card. Máxima} \\
\hline
cd\_linha & Simples, determinante & 1 & 1 \\
nm\_linha & Simples & 1 & 1 \\
cd\_site & Simples, estrangeiro & 1 & 1 \\
\hline
\end{tabular}
\caption{Entidade LINHA\_PRODUCAO e seus respectivos atributos, tipos e cardinalidade.}
\label{tab:linha_producao}
\end{table}

\subsection*{ENTIDADE: MAQUINA}
Equipamento monitorado pelo gêmeo digital (xDT).

\begin{table}[h!]
\centering
\begin{tabular}{| l | l | c | c |}
\hline
\rowcolor{red}\color{white}\textbf{Atributo} & \color{white}\textbf{Tipo do Atributo} & \color{white}\textbf{Card. Mínima} & \color{white}\textbf{Card. Máxima} \\
\hline
cd\_maquina & Simples, determinante & 1 & 1 \\
nm\_maquina & Simples & 1 & 1 \\
tp\_maquina & Simples & 1 & 1 \\
nm\_fabricante & Simples & 0 & 1 \\
nm\_modelo & Simples & 0 & 1 \\
cd\_linha & Simples, estrangeiro & 1 & 1 \\
\hline
\end{tabular}
\caption{Entidade MAQUINA e seus respectivos atributos, tipos e cardinalidade.}
\label{tab:maquina}
\end{table}

\subsection*{ENTIDADE: TIPO\_SENSOR}
Catálogo de tipos/unidades e faixas operacionais.

\begin{table}[h!]
\centering
\begin{tabular}{| l | l | c | c |}
\hline
\rowcolor{red}\color{white}\textbf{Atributo} & \color{white}\textbf{Tipo do Atributo} & \color{white}\textbf{Card. Mínima} & \color{white}\textbf{Card. Máxima} \\
\hline
cd\_tipo\_sensor & Simples, determinante & 1 & 1 \\
ds\_tipo & Simples & 1 & 1 \\
unidade & Simples & 1 & 1 \\
vl\_min & Simples & 0 & 1 \\
vl\_max & Simples & 0 & 1 \\
\hline
\end{tabular}
\caption{Entidade TIPO\_SENSOR e seus respectivos atributos, tipos e cardinalidade.}
\label{tab:tipo_sensor}
\end{table}

\subsection*{ENTIDADE: SENSOR}
Dispositivo físico associado a uma máquina.

\begin{table}[h!]
\centering
\begin{tabular}{| l | l | c | c |}
\hline
\rowcolor{red}\color{white}\textbf{Atributo} & \color{white}\textbf{Tipo do Atributo} & \color{white}\textbf{Card. Mínima} & \color{white}\textbf{Card. Máxima} \\
\hline
cd\_sensor & Simples, determinante & 1 & 1 \\
tp\_sensor & Simples & 1 & 1 \\
nm\_modelo & Simples & 0 & 1 \\
unidade & Simples & 1 & 1 \\
cd\_tipo\_sensor & Simples, estrangeiro & 1 & 1 \\
cd\_maquina & Simples, estrangeiro & 1 & 1 \\
\hline
\end{tabular}
\caption{Entidade SENSOR e seus respectivos atributos, tipos e cardinalidade.}
\label{tab:sensor}
\end{table}

\subsection*{ENTIDADE: TOPICO\_MQTT}
Tópicos de publicação/assinatura.

\begin{table}[h!]
\centering
\begin{tabular}{| l | l | c | c |}
\hline
\rowcolor{red}\color{white}\textbf{Atributo} & \color{white}\textbf{Tipo do Atributo} & \color{white}\textbf{Card. Mínima} & \color{white}\textbf{Card. Máxima} \\
\hline
cd\_topico & Simples, determinante & 1 & 1 \\
nm\_topico & Simples & 1 & 1 \\
ds\_schema & Simples & 0 & 1 \\
\hline
\end{tabular}
\caption{Entidade TOPICO\_MQTT e seus respectivos atributos, tipos e cardinalidade.}
\label{tab:topico_mqtt}
\end{table}

\subsection*{ENTIDADE: MENSAGEM\_MQTT}
Envelope técnico recebido/enviado no broker.

\begin{table}[h!]
\centering
\begin{tabular}{| l | l | c | c |}
\hline
\rowcolor{red}\color{white}\textbf{Atributo} & \color{white}\textbf{Tipo do Atributo} & \color{white}\textbf{Card. Mínima} & \color{white}\textbf{Card. Máxima} \\
\hline
cd\_msg & Simples, determinante & 1 & 1 \\
cd\_topico & Simples, estrangeiro & 1 & 1 \\
ts\_evento & Simples & 1 & 1 \\
payload\_json & Simples & 1 & 1 \\
\hline
\end{tabular}
\caption{Entidade MENSAGEM\_MQTT e seus respectivos atributos, tipos e cardinalidade.}
\label{tab:mensagem_mqtt}
\end{table}

\subsection*{ENTIDADE: LEITURA\_SENSOR}
Snapshot relacional de séries temporais (espelho do InfluxDB).

\begin{table}[h!]
\centering
\begin{tabular}{| l | l | c | c |}
\hline
\rowcolor{red}\color{white}\textbf{Atributo} & \color{white}\textbf{Tipo do Atributo} & \color{white}\textbf{Card. Mínima} & \color{white}\textbf{Card. Máxima} \\
\hline
cd\_leitura & Simples, determinante & 1 & 1 \\
cd\_sensor & Simples, estrangeiro & 1 & 1 \\
ts\_leitura & Simples & 1 & 1 \\
vl\_medido & Simples & 1 & 1 \\
cd\_msg & Simples, estrangeiro & 0 & 1 \\
\hline
\end{tabular}
\caption{Entidade LEITURA\_SENSOR e seus respectivos atributos, tipos e cardinalidade.}
\label{tab:leitura_sensor}
\end{table}

\subsection*{ENTIDADE: ANOMALIA}
Eventos detectados (outlier, drift, padrão de falha).

\begin{table}[h!]
\centering
\begin{tabular}{| l | l | c | c |}
\hline
\rowcolor{red}\color{white}\textbf{Atributo} & \color{white}\textbf{Tipo do Atributo} & \color{white}\textbf{Card. Mínima} & \color{white}\textbf{Card. Máxima} \\
\hline
cd\_anomalia & Simples, determinante & 1 & 1 \\
tp\_anomalia & Simples & 1 & 1 \\
score & Simples & 1 & 1 \\
gravidade & Simples & 1 & 1 \\
cd\_sensor & Simples, estrangeiro & 1 & 1 \\
ts\_inicio & Simples & 1 & 1 \\
ts\_fim & Simples & 0 & 1 \\
\hline
\end{tabular}
\caption{Entidade ANOMALIA e seus respectivos atributos, tipos e cardinalidade.}
\label{tab:anomalia}
\end{table}

\subsection*{ENTIDADE: ORDEM\_MANUTENCAO}
Integração ERP/MES/CMMS.
\begin{center}
\begin{tabular}{| l | l | c | c |}
\hline
\rowcolor{red}\color{white}\textbf{Atributo} & \color{white}\textbf{Tipo do Atributo} & \color{white}\textbf{Card. Mínima} & \color{white}\textbf{Card. Máxima} \\
\hline
cd\_os & Simples, determinante & 1 & 1 \\
num\_os\_ext & Simples & 0 & 1 \\
tp\_os & Simples & 1 & 1 \\
st\_os & Simples & 1 & 1 \\
ts\_abertura & Simples & 1 & 1 \\
ts\_fechamento & Simples & 0 & 1 \\
cd\_alerta & Simples, estrangeiro & 0 & 1 \\
cd\_maquina & Simples, estrangeiro & 1 & 1 \\
\hline
\end{tabular}
\end{center}

\subsection*{ENTIDADE: ACAO\_MANUTENCAO}
Passos executados numa OS.
\begin{center}
\begin{tabular}{| l | l | c | c |}
\hline
\rowcolor{red}\color{white}\textbf{Atributo} & \color{white}\textbf{Tipo do Atributo} & \color{white}\textbf{Card. Mínima} & \color{white}\textbf{Card. Máxima} \\
\hline
cd\_acao & Simples, determinante & 1 & 1 \\
cd\_os & Simples, estrangeiro & 1 & 1 \\
ds\_acao & Simples & 1 & 1 \\
responsavel\_exec & Simples & 0 & 1 \\
ts\_inicio & Simples & 1 & 1 \\
ts\_fim & Simples & 0 & 1 \\
custo\_real & Simples & 0 & 1 \\
\hline
\end{tabular}
\end{center}

\subsection*{ENTIDADE: EVENTO\_PARADA}
Downtime de máquinas.
\begin{center}
\begin{tabular}{| l | l | c | c |}
\hline
\rowcolor{red}\color{white}\textbf{Atributo} & \color{white}\textbf{Tipo do Atributo} & \color{white}\textbf{Card. Mínima} & \color{white}\textbf{Card. Máxima} \\
\hline
cd\_parada & Simples, determinante & 1 & 1 \\
cd\_maquina & Simples, estrangeiro & 1 & 1 \\
ts\_inicio & Simples & 1 & 1 \\
ts\_fim & Simples & 1 & 1 \\
motivo & Simples & 0 & 1 \\
duracao\_min & Simples & 1 & 1 \\
\hline
\end{tabular}
\end{center}

\subsection*{ENTIDADE: KPI\_OEE}
Indicadores agregados por período.

\begin{table}[h!]
\centering
\begin{tabular}{| l | l | c | c |}
\hline
\rowcolor{red}\color{white}\textbf{Atributo} & \color{white}\textbf{Tipo do Atributo} & \color{white}\textbf{Card. Mínima} & \color{white}\textbf{Card. Máxima} \\
\hline
cd\_kpi & Simples, determinante & 1 & 1 \\
cd\_maquina & Simples, estrangeiro & 1 & 1 \\
dt\_ref & Simples & 1 & 1 \\
disponibilidade & Simples & 1 & 1 \\
performance & Simples & 1 & 1 \\
qualidade & Simples & 1 & 1 \\
oee & Simples & 1 & 1 \\
\hline
\end{tabular}
\caption{Entidade KPI\_OEE e seus respectivos atributos, tipos e cardinalidade.}
\label{tab:kpi_oee}
\end{table}

\subsection*{ENTIDADE: CUSTO\_EVIDADO}
Registro financeiro de perdas evitadas (ROI em tempo real).

\begin{table}[h!]
\centering
\begin{tabular}{| l | l | c | c |}
\hline
\rowcolor{red}\color{white}\textbf{Atributo} & \color{white}\textbf{Tipo do Atributo} & \color{white}\textbf{Card. Mínima} & \color{white}\textbf{Card. Máxima} \\
\hline
cd\_custo & Simples, determinante & 1 & 1 \\
cd\_alerta & Simples, estrangeiro & 0 & 1 \\
cd\_acao & Simples, estrangeiro & 0 & 1 \\
estimativa\_valor & Simples & 1 & 1 \\
metodologia & Simples & 1 & 1 \\
moeda & Simples & 1 & 1 \\
ts\_registro & Simples & 1 & 1 \\
\hline
\end{tabular}
\caption{Entidade CUSTO\_EVIDADO e seus respectivos atributos, tipos e cardinalidade.}
\label{tab:custo_evitado}
\end{table}

\subsection*{ENTIDADE: USUARIO}
Usuários/operadores para uso e gamificação.

\begin{table}[h!]
\centering
\begin{tabular}{| l | l | c | c |}
\hline
\rowcolor{red}\color{white}\textbf{Atributo} & \color{white}\textbf{Tipo do Atributo} & \color{white}\textbf{Card. Mínima} & \color{white}\textbf{Card. Máxima} \\
\hline
cd\_usuario & Simples, determinante & 1 & 1 \\
nm\_usuario & Simples & 1 & 1 \\
perfil & Simples & 1 & 1 \\
email & Simples & 1 & 1 \\
\hline
\end{tabular}
\caption{Entidade USUARIO e seus respectivos atributos, tipos e cardinalidade.}
\label{tab:usuario}
\end{table}

\subsection*{ENTIDADE: GAMIFICACAO\_EVENTO}
Pontuação e recompensas por ações.

\begin{table}[h!]
\centering
\begin{tabular}{| l | l | c | c |}
\hline
\rowcolor{red}\color{white}\textbf{Atributo} & \color{white}\textbf{Tipo do Atributo} & \color{white}\textbf{Card. Mínima} & \color{white}\textbf{Card. Máxima} \\
\hline
cd\_evento\_gam & Simples, determinante & 1 & 1 \\
cd\_usuario & Simples, estrangeiro & 1 & 1 \\
tp\_evento & Simples & 1 & 1 \\
pontos & Simples & 1 & 1 \\
ts\_evento & Simples & 1 & 1 \\
cd\_alerta & Simples, estrangeiro & 0 & 1 \\
\hline
\end{tabular}
\caption{Entidade GAMIFICACAO\_EVENTO e seus respectivos atributos, tipos e cardinalidade.}
\label{tab:gamificacao_evento}
\end{table}

\subsection*{ENTIDADE: GEMEO\_DIGITAL\_MODELO}
Versões do modelo físico-informado/simulação (xDT).

\begin{table}[h!]
\centering
\begin{tabular}{| l | l | c | c |}
\hline
\rowcolor{red}\color{white}\textbf{Atributo} & \color{white}\textbf{Tipo do Atributo} & \color{white}\textbf{Card. Mínima} & \color{white}\textbf{Card. Máxima} \\
\hline
cd\_modelo & Simples, determinante & 1 & 1 \\
versao & Simples & 1 & 1 \\
tp\_modelo & Simples & 1 & 1 \\
ds\_hiperparam & Simples & 0 & 1 \\
\hline
\end{tabular}
\caption{Entidade GEMEO\_DIGITAL\_MODELO e seus respectivos atributos, tipos e cardinalidade.}
\label{tab:gemeo_digital_modelo}
\end{table}

\subsection*{ENTIDADE: CENARIO\_SIMULACAO}
Cenários what-if executados no xDT.

\begin{table}[h!]
\centering
\begin{tabular}{| l | l | c | c |}
\hline
\rowcolor{red}\color{white}\textbf{Atributo} & \color{white}\textbf{Tipo do Atributo} & \color{white}\textbf{Card. Mínima} & \color{white}\textbf{Card. Máxima} \\
\hline
cd\_cenario & Simples, determinante & 1 & 1 \\
cd\_modelo & Simples, estrangeiro & 1 & 1 \\
cd\_maquina & Simples, estrangeiro & 1 & 1 \\
nm\_cenario & Simples & 1 & 1 \\
parametros\_json & Simples & 1 & 1 \\
ts\_execucao & Simples & 1 & 1 \\
resultado\_json & Simples & 1 & 1 \\
\hline
\end{tabular}
\caption{Entidade CENARIO\_SIMULACAO e seus respectivos atributos, tipos e cardinalidade.}
\label{tab:cenario_simulacao}
\end{table}

\subsection*{ENTIDADE: POLITICA\_RL}
Políticas de aprendizado por reforço.

\begin{table}[h!]
\centering
\begin{tabular}{| l | l | c | c |}
\hline
\rowcolor{red}\color{white}\textbf{Atributo} & \color{white}\textbf{Tipo do Atributo} & \color{white}\textbf{Card. Mínima} & \color{white}\textbf{Card. Máxima} \\
\hline
cd\_politica & Simples, determinante & 1 & 1 \\
algoritmo & Simples & 1 & 1 \\
versao & Simples & 1 & 1 \\
recompensa\_media & Simples & 0 & 1 \\
cd\_modelo & Simples, estrangeiro & 1 & 1 \\
st\_implantada & Simples & 1 & 1 \\
\hline
\end{tabular}
\caption{Entidade POLITICA\_RL e seus respectivos atributos, tipos e cardinalidade.}
\label{tab:politica_rl}
\end{table}

\subsection*{ENTIDADE: DATASET}
Conjuntos de dados usados em ML/IA generativa.

\begin{table}[h!]
\centering
\begin{tabular}{| l | l | c | c |}
\hline
\rowcolor{red}\color{white}\textbf{Atributo} & \color{white}\textbf{Tipo do Atributo} & \color{white}\textbf{Card. Mínima} & \color{white}\textbf{Card. Máxima} \\
\hline
cd\_dataset & Simples, determinante & 1 & 1 \\
nm\_dataset & Simples & 1 & 1 \\
fonte & Simples & 1 & 1 \\
num\_registros & Simples & 0 & 1 \\
caminho\_arquivo & Simples & 1 & 1 \\
\hline
\end{tabular}
\caption{Entidade DATASET e seus respectivos atributos, tipos e cardinalidade.}
\label{tab:dataset}
\end{table}

\subsection*{ENTIDADE: TREINO\_MODELO}
Rastreamento de experimentos/artefatos de ML.

\begin{table}[h!]
\centering
\begin{tabular}{| l | l | c | c |}
\hline
\rowcolor{red}\color{white}\textbf{Atributo} & \color{white}\textbf{Tipo do Atributo} & \color{white}\textbf{Card. Mínima} & \color{white}\textbf{Card. Máxima} \\
\hline
cd\_treino & Simples, determinante & 1 & 1 \\
cd\_dataset & Simples, estrangeiro & 1 & 1 \\
tp\_problema & Simples & 1 & 1 \\
algoritmo & Simples & 1 & 1 \\
metricas\_json & Simples & 1 & 1 \\
artefato\_uri & Simples & 1 & 1 \\
\hline
\end{tabular}
\caption{Entidade TREINO\_MODELO e seus respectivos atributos, tipos e cardinalidade.}
\label{tab:treino_modelo}
\end{table}

\newpage
% ==============================
\section*{Relacionamentos}
\small
\subsection*{SITE — LINHA\_PRODUCAO}
Uma site possui várias linhas; cada linha pertence a um site.
\begin{center}
\begin{tabular}{|l|c|c|}
\hline
\rowcolor{vermelhoheader}\color{white}\textbf{Entidade} & \textbf{Card. Mínima} & \textbf{Card. Máxima} \\
\hline
SITE & 1 & N \\
LINHA\_PRODUCAO & 1 & 1 \\
\hline
\end{tabular}
\end{center}
\hfill
\subsection*{LINHA\_PRODUCAO — MAQUINA}
Uma linha tem várias máquinas; cada máquina pertence a uma linha.
\begin{center}
\begin{tabular}{|l|c|c|}
\hline
\rowcolor{vermelhoheader}\color{white}\textbf{Entidade} & \textbf{Card. Mínima} & \textbf{Card. Máxima} \\
\hline
LINHA\_PRODUCAO & 1 & N \\
MAQUINA & 1 & 1 \\
\hline
\end{tabular}
\end{center}
\vspace{0.5cm}

\subsection*{MAQUINA — SENSOR}
Uma máquina pode ter muitos sensores; cada sensor pertence a uma máquina.
\begin{center}
\begin{tabular}{|l|c|c|}
\hline
\rowcolor{vermelhoheader}\color{white}\textbf{Entidade} & \textbf{Card. Mínima} & \textbf{Card. Máxima} \\
\hline
MAQUINA & 1 & N \\
SENSOR & 1 & 1 \\
\hline
\end{tabular}
\end{center}
\hfill
\subsection*{TIPO\_SENSOR — SENSOR}
Um tipo cataloga muitos sensores; cada sensor referencia um tipo.
\begin{center}
\begin{tabular}{|l|c|c|}
\hline
\rowcolor{vermelhoheader}\color{white}\textbf{Entidade} & \textbf{Card. Mínima} & \textbf{Card. Máxima} \\
\hline
TIPO\_SENSOR & 1 & N \\
SENSOR & 1 & 1 \\
\hline
\end{tabular}
\end{center}
\vspace{0.5cm}

\subsection*{TOPICO\_MQTT — MENSAGEM\_MQTT}
Um tópico recebe muitas mensagens; cada mensagem pertence a um tópico.
\begin{center}
\begin{tabular}{|l|c|c|}
\hline
\rowcolor{vermelhoheader}\color{white}\textbf{Entidade} & \textbf{Card. Mínima} & \textbf{Card. Máxima} \\
\hline
TOPICO\_MQTT & 1 & N \\
MENSAGEM\_MQTT & 1 & 1 \\
\hline
\end{tabular}
\end{center}
\hfill
\subsection*{MENSAGEM\_MQTT — LEITURA\_SENSOR}
Uma mensagem pode conter múltiplas leituras; cada leitura pode referenciar a mensagem de origem.
\begin{center}
\begin{tabular}{|l|c|c|}
\hline
\rowcolor{vermelhoheader}\color{white}\textbf{Entidade} & \textbf{Card. Mínima} & \textbf{Card. Máxima} \\
\hline
MENSAGEM\_MQTT & 0 & N \\
LEITURA\_SENSOR & 0 & 1 \\
\hline
\end{tabular}
\end{center}
\vspace{0.5cm}

\subsection*{SENSOR — LEITURA\_SENSOR}
Um sensor gera muitas leituras; cada leitura é de um sensor.
\begin{center}
\begin{tabular}{|l|c|c|}
\hline
\rowcolor{vermelhoheader}\color{white}\textbf{Entidade} & \textbf{Card. Mínima} & \textbf{Card. Máxima} \\
\hline
SENSOR & 1 & N \\
LEITURA\_SENSOR & 1 & 1 \\
\hline
\end{tabular}
\end{center}
\hfill
\subsection*{SENSOR — ANOMALIA}
Um sensor pode originar várias anomalias; cada anomalia é detectada em um sensor.
\begin{center}
\begin{tabular}{|l|c|c|}
\hline
\rowcolor{vermelhoheader}\color{white}\textbf{Entidade} & \textbf{Card. Mínima} & \textbf{Card. Máxima} \\
\hline
SENSOR & 1 & N \\
ANOMALIA & 1 & 1 \\
\hline
\end{tabular}
\end{center}
\vspace{0.5cm}

\subsection*{ANOMALIA — ALERTA}
Uma anomalia pode disparar múltiplos alertas; cada alerta referencia uma anomalia.
\begin{center}
\begin{tabular}{|l|c|c|}
\hline
\rowcolor{vermelhoheader}\color{white}\textbf{Entidade} & \textbf{Card. Mínima} & \textbf{Card. Máxima} \\
\hline
ANOMALIA & 1 & N \\
ALERTA & 1 & 1 \\
\hline
\end{tabular}
\end{center}
\hfill
\subsection*{ALERTA — ORDEM\_MANUTENCAO}
Um alerta pode (ou não) abrir uma OS; uma OS geralmente se origina de 1 alerta.
\begin{center}
\begin{tabular}{|l|c|c|}
\hline
\rowcolor{vermelhoheader}\color{white}\textbf{Entidade} & \textbf{Card. Mínima} & \textbf{Card. Máxima} \\
\hline
ALERTA & 0 & 1 \\
ORDEM\_MANUTENCAO & 1 & 1 \\
\hline
\end{tabular}
\end{center}
\vspace{0.5cm}

\subsection*{ORDEM\_MANUTENCAO — ACAO\_MANUTENCAO}
Uma OS possui várias ações; cada ação pertence a uma OS.
\begin{center}
\begin{tabular}{|l|c|c|}
\hline
\rowcolor{vermelhoheader}\color{white}\textbf{Entidade} & \textbf{Card. Mínima} & \textbf{Card. Máxima} \\
\hline
ORDEM\_MANUTENCAO & 1 & N \\
ACAO\_MANUTENCAO & 1 & 1 \\
\hline
\end{tabular}
\end{center}
\hfill
\subsection*{MAQUINA — EVENTO\_PARADA}
Uma máquina pode ter várias paradas; cada parada pertence a uma máquina.
\begin{center}
\begin{tabular}{|l|c|c|}
\hline
\rowcolor{vermelhoheader}\color{white}\textbf{Entidade} & \textbf{Card. Mínima} & \textbf{Card. Máxima} \\
\hline
MAQUINA & 1 & N \\
EVENTO\_PARADA & 1 & 1 \\
\hline
\end{tabular}
\end{center}
\vspace{0.5cm}

\subsection*{MAQUINA — KPI\_OEE}
Indicadores agregados por máquina/período.
\begin{center}
\begin{tabular}{|l|c|c|}
\hline
\rowcolor{vermelhoheader}\color{white}\textbf{Entidade} & \textbf{Card. Mínima} & \textbf{Card. Máxima} \\
\hline
MAQUINA & 1 & N \\
KPI\_OEE & 1 & 1 \\
\hline
\end{tabular}
\end{center}
\hfill
\subsection*{ALERTA — CUSTO\_EVIDADO}
Custos evitados podem ser atribuídos ao alerta.
\begin{center}
\begin{tabular}{|l|c|c|}
\hline
\rowcolor{vermelhoheader}\color{white}\textbf{Entidade} & \textbf{Card. Mínima} & \textbf{Card. Máxima} \\
\hline
ALERTA & 0 & N \\
CUSTO\_EVIDADO & 1 & 1 \\
\hline
\end{tabular}
\end{center}
\vspace{0.5cm}

\subsection*{ACAO\_MANUTENCAO — CUSTO\_EVIDADO}
Custos evitados podem ser atribuídos à ação.
\begin{center}
\begin{tabular}{|l|c|c|}
\hline
\rowcolor{vermelhoheader}\color{white}\textbf{Entidade} & \textbf{Card. Mínima} & \textbf{Card. Máxima} \\
\hline
ACAO\_MANUTENCAO & 0 & N \\
CUSTO\_EVIDADO & 1 & 1 \\
\hline
\end{tabular}
\end{center}
\hfill
\subsection*{USUARIO — GAMIFICACAO\_EVENTO}
Um usuário recebe vários eventos/pontos; cada evento pertence a um usuário.
\begin{center}
\begin{tabular}{|l|c|c|}
\hline
\rowcolor{vermelhoheader}\color{white}\textbf{Entidade} & \textbf{Card. Mínima} & \textbf{Card. Máxima} \\
\hline
USUARIO & 1 & N \\
GAMIFICACAO\_EVENTO & 1 & 1 \\
\hline
\end{tabular}
\end{center}
\vspace{0.5cm}

\subsection*{GEMEO\_DIGITAL\_MODELO — CENARIO\_SIMULACAO}
Um modelo (versão) gera vários cenários; cada cenário referencia um modelo.
\begin{center}
\begin{tabular}{|l|c|c|}
\hline
\rowcolor{vermelhoheader}\color{white}\textbf{Entidade} & \textbf{Card. Mínima} & \textbf{Card. Máxima} \\
\hline
GEMEO\_DIGITAL\_MODELO & 1 & N \\
CENARIO\_SIMULACAO & 1 & 1 \\
\hline
\end{tabular}
\end{center}
\hfill
\subsection*{GEMEO\_DIGITAL\_MODELO — POLITICA\_RL}
Uma política RL é treinada/associada a uma versão do xDT; várias políticas por modelo.
\begin{center}
\begin{tabular}{|l|c|c|}
\hline
\rowcolor{vermelhoheader}\color{white}\textbf{Entidade} & \textbf{Card. Mínima} & \textbf{Card. Máxima} \\
\hline
GEMEO\_DIGITAL\_MODELO & 1 & N \\
POLITICA\_RL & 1 & 1 \\
\hline
\end{tabular}
\end{center}
\vspace{0.5cm}

\subsection*{MAQUINA — CENARIO\_SIMULACAO}
Cenários what-if executados para uma máquina específica.
\begin{center}
\begin{tabular}{|l|c|c|}
\hline
\rowcolor{vermelhoheader}\color{white}\textbf{Entidade} & \textbf{Card. Mínima} & \textbf{Card. Máxima} \\
\hline
MAQUINA & 1 & N \\
CENARIO\_SIMULACAO & 1 & 1 \\
\hline
\end{tabular}
\end{center}
\hfill
\subsection*{DATASET — TREINO\_MODELO}
Um dataset alimenta vários treinamentos; cada treino referencia um dataset.
\begin{center}
\begin{tabular}{|l|c|c|}
\hline
\rowcolor{vermelhoheader}\color{white}\textbf{Entidade} & \textbf{Card. Mínima} & \textbf{Card. Máxima} \\
\hline
DATASET & 1 & N \\
TREINO\_MODELO & 1 & 1 \\
\hline
\end{tabular}
\end{center}


\end{document}
